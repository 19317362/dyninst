\subsection{Graph}
\definedin{common/h/Graph.h}
\label{sec:graph}

We provide a generic graph interface, which allows users adding, deleting,
iterating nodes and edges in a graph. Our slicing algorithms are implemented
upon this graph interface, so users can inherit the defined classes for
customization.

\begin{apient}
typedef boost::shared_ptr<Graph> Graph::Ptr;
\end{apient}
\apidesc{Shared pointer for Graph}

\begin{apient}
virtual void Graph::entryNodes(NodeIterator &begin, NodeIterator &end);
\end{apient}
\apidesc{The entry nodes (nodes without any incoming edges) of the graph.}

    // If you want to traverse the graph backwards start here.
\begin{apient}
virtual void Graph::exitNodes(NodeIterator &begin, NodeIterator &end);
\end{apient}
\apidesc{The exit nodes (nodes without any outgoing edges) of the graph.}
    
\begin{apient}
virtual void Graph::allNodes(NodeIterator &begin, NodeIterator &end);
\end{apient}
\apidesc{Iterate all nodes in the graph.}


\begin{apient}
class Graph::NodePredicate;
typedef boost::shared_ptr<Graph::NodePredicate> Graph::NodePredicate::Ptr;
virtual bool Graph::NodePredicate::predicate(const NodePtr &node) = 0;

\end{apient}
\apidesc{Interface class for predicate-based searches. Users can inherit this
class to specify the functor to use as a predicate to iterate over nodes in the
graph.}


\begin{apient}
virtual bool Graph::find(Address addr, NodeIterator &begin, NodeIterator &end);

typedef bool (*NodePredicateFunc)(const NodePtr &node, void *user_arg);
virtual bool Graph::find(NodePredicate::Ptr, NodeIterator &begin, NodeIterator &end);

virtual bool Graph::find(NodePredicateFunc, void *user_arg, NodeIterator &begin, NodeIterator &end);
\end{apient}
\apidesc{Find all nodes satisfying the given conditions} 

\begin{apient}
bool Graph::printDOT(const std::string& fileName);
\end{apient}
\apidesc{Output the graph in dot format.}

\begin{apient}
static Graph::Ptr Graph::createGraph();
\end{apient}
\apidesc{Return an empty graph.}

\begin{apient}
void Graph::insertPair(NodePtr source, NodePtr target, EdgePtr edge = EdgePtr());
\end{apient}
\apidesc{Insert an pair of node into the graph and create a new edge \code{edge} from
\code{source} to \code{target}.}

\begin{apient}
virtual void Graph::insertEntryNode(NodePtr entry);
virtual void Graph::insertExitNode(NodePtr exit);
\end{apient}
\apidesc{Insert a node as an entry/exit node}

\begin{apient}
virtual void Graph::markAsEntryNode(NodePtr entry);
virtual void Graph::markAsExitNode(NodePtr exit);
\apidesc{Mark a node that has been added to this graph as an entry/exit node.}


\begin{apient}
void Graph::deleteNode(NodePtr node);
void Graph::addNode(NodePtr node);
\end{apient}
\apidesc{Delete / Add a node.}

\begin{apient}
bool Graph::isEntryNode(NodePtr node);
bool Graph::isExitNode(NodePtr node);
\end{apient}
\apidesc{Check whether a node is an entry / exit node}

\begin{apient}
void Graph::clearEntryNodes();
void Graph::clearExitNodes();
\end{apient}
\apidesc{Clear the marking of entry / exit nodes. Note that the nodes are not
deleted from the graph.}

\begin{apient}
unsigned Graph::size() const;
\end{apient}
\apidesc{Return the number of nodes in the graph.}

\subsection{Node}
\definedin{common/h/Node.h}

\begin{apient}
typedef boost::shared_ptr<Node> Node::Ptr;
\end{apient}
\apidesc{shared pointer for Node}

\begin{apient}
void Node::ins(EdgeIterator &begin, EdgeIterator &end);
void Node::outs(EdgeIterator &begin, EdgeIterator &end);
\end{apient}
\apidesc{Iterate over incoming/outgoing edges of this node.}

\begin{apient}
void Node::ins(NodeIterator &begin, NodeIterator &end);
void Node::outs(NodeIterator &begin, NodeIterator &end);
\end{apient}
\apidesc{Iterate over adjacent nodes connected with incoming/outgoing edges of
this node}.

\begin{apient}
bool Node::hasInEdges(); 
bool Node::hasOutEdges();
\end{apient}
\apidesc{Return \code{true} if this node has incoming/outgoing edges.}

\begin{apient}
void Node::deleteInEdge(EdgeIterator e);
void Node::deleteOutEdge(EdgeIterator e);
\end{apient}
\apidesc{Delete an incoming/outgoing edge.}

\begin{apient}
virtual Address Node::addr() const;
\end{apient}
\apidesc{Return the address of this node.}
    
\begin{apient}    
virtual std::string Node::format() const = 0;
\end{apient}
\apidesc{Return the string representation.}

\begin{apient}
class NodeIterator
\end{apient}
\apidesc{Iterator for nodes. Common iterator operations including \code{++},
\code{--}, and dereferencing are supported.}


\subsection{Edge}
\definedin{common/h/Edge.h}

\begin{apient}
typedef boost::shared_ptr<Edge> Edge::Ptr;
\end{apient}
\apidesc{Shared pointer for \code{Edge}.}

\begin{apient}
static Edge::Ptr Edge::createEdge(const Node::Ptr source, const Node::Ptr target);
\end{apient}
\apidesc{Create a new directed edge from \code{source} to \code{target}.}

\begin{apient}
Node::Ptr Edge::source() const; 
Node::Ptr Edge::target() const;
\end{apient}
\apidesc{Return the source / target node.}

\begin{apient}
void Edge::setSource(Node::Ptr source); 
void Edge::setTarget(Node::Ptr target);
\end{apient}
\apidesc{Set the source / target node.}

\begin{apient}
class EdgeIterator
\end{apient}
\apidesc{Iterator for edges. Common iterator operations including \code{++},
\code{--}, and dereferencing are supported.}

