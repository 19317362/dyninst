\section{Abstractions}
\label{sec:abstractions}

DataflowAPI starts from the control flow graphs generated by ParseAPI and the instructions generated by InstructionAPI.
From these, it provides dataflow facts in a variety of forms. The key abstractions used by DataflowAPI are:

\begin{itemize}
\item[Abstract Location] Represents a register or memory location (cite).
DataflowAPI provides three types of abstract locations: registers, stack, and
help. Register abstract locations are distinguished through register names.
Stack abstract locations are distinguished by the offset within a stack frame
and the function to which the stack frame belongs.
Head abstract locations are distinguished by the virtual address of the
location.
\item[Abstract Region] Represents a set of abstract locations of the same type
(cite). If an abstract region contains only a single abstract location, the
abstract location is precisely represented. 
If an abstract region contains more than one abstract locations, the region
contains the type of the locations. For a memory region, it also contains 
the memory address calculation that gives rise to this region. The address
calculation is represented with an AST described below.
\item[AST] Represents the symbolic expansion of an instruction's full semantics
\item[Semantics Policy] Provides callbacks for each low-level operation an instruction may perform.
\item[Assignment] Represents a single data dependency of abstract regions in an instruction. For example, xchg eax, ebx creates two assignments: one from pre-instruction eax to post-instruction ebx, and one from pre-instruction ebx to post-instruction eax.
\item[Stack Height] Represents the difference between a value in an abstract location and the stack pointer at a function's call site.
\item[Graph] Represents a directed graph.
\item[Symbolic Evaluator] Translates a graph representing a program slice (cite) into a mapping from assignments to ASTs.
\end{itemize}


